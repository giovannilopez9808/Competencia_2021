\hrule
\vspace{0.5cm}
\section{Logística}
\subsection{Fechas de recepción de datos}
Las fechas propuestas para la realización de la competencia de natación es sábado 4 y domingo 5 de septiembre del 2021. Estas fechas son seleccionadas para dar tiempo a recibir la información de los nadadores antes del día lunes 30 de agosto del 2021 para confirmar que los datos sean correctos. Si se encuentran errores en los datos, se interpretarán para que sigan el formato antes mencionado y se le enviará un correo al responsale comentando los errores identificados. El responsable deberá confirmar que los datos interpretados son correctos.

\subsection{Organización}
\subsubsection{Afloje}
Cada organización tendra 15 minutos de afloje en la alberca. Los carriles dependera de la cantidad de nadadores en cada organización.

\subsection{Pruebas}
El primer día se nadarán la prueba con mayor y menor densidad de nadadores para la categoria de 8 y menores, el segundo día se nadarán las dos pruebas restantes. En el caso de las categorias de 9 y mayores, el primer día se nadarán las pruebas que se encuentren en el primer, segundo, penultimo y ultimo lugar con respecto a su densidad de nadadores en cada categoria. El segundo día se nadarán las pruebas restantes.
\subsubsection{Ejemplo}
Supongamos que el orden de mayor a menor densidad de las prubeas para cada categoria son las mostradas en la tabla \ref{table:orden_example}.
\begin{table}[H]
    \centering
    \begin{tabular}{ccc}
        \hline
        Orden & 8 y menores & 9 y menores \\ \hline
        1     & 25M         & 50D         \\
        2     & 25D         & 100L        \\
        3     & 25L         & 50P         \\
        4     & 25P         & 100CI       \\
        5     & -           & 50L         \\
        6     & -           & 100P        \\
        7     & -           & 100D        \\
        8     & -           & 50M         \\ \hline
    \end{tabular}
    \caption{Ejemplo de orden de las pruebas}
    \label{table:orden_example}
\end{table}
Entonces el orden de las pruebas será el siguiente para los dos días sera el siguiente:
\begin{table}[H]
    \centering
    \begin{tabular}{cccc}
        \hline
        Categoria                    & Pruebas & Día 1 & Día 2 \\ \hline
        \multirow{2}{*}{8 y menores} & 1       & 25M   & 25D   \\
                                     & 2       & 25P   & 25L   \\ \hline
        \multirow{4}{*}{9 y mayores} & 1       & 50D   & 50P   \\
                                     & 2       & 100L  & 100CI \\
                                     & 3       & 100D  & 50L   \\
                                     & 4       & 50M   & 100D  \\ \hline
    \end{tabular}
    \caption{Orden de las pruebas en cada día usando el caso de la tabla \ref{table:orden_example}}
\end{table}
Al contar con un espacio reducido se proponen las siguientes organizaciones:



\subsubsection{Organización tipo 1}
La competencia iniciará con la primer prueba de la categoria de 8 y menores, en seguida empezará la primer prueba de 9 y mayores. Al termino se nadará la segunda prueba de la categoria 9 y mayores para después comenzar con la segunda prueba de 8 y menores. Para terminar el día con la tercer y cuarta prueba del primer día de 9 y mayores. El segundo día se iniciará la misma dinámica con las pruebas correspondientes. Usando el caso de la tabla \ref{table:orden_example}, el orden de cada prueba sería el siguiente:
\begin{table}[H]
    \centering
    \begin{tabular}{ccc}
        \hline
        Orden del día & Día 1              & Día 2               \\  \hline
        1             & 8 y menores - 25M  & 8 y menores - 25D   \\
        2             & 9 y mayores - 50P  & 9 y mayores - 50P   \\
        3             & 9 y mayores - 100L & 9 y mayores - 100CI \\
        4             & 8 y menores - 25P  & 8 y menores - 25L   \\
        5             & 9 y mayores - 100D & 9 y mayores - 50L   \\
        6             & 9 y mayores - 50M  & 9 y mayores - 100D  \\ \hline
    \end{tabular}
    \caption{Organización de la pruebas siguiendo la organización 1 y el caso de la tabla \ref{table:orden_example}.}
    \label{table:organization1}
\end{table}

\subsubsection{Organización tipo 2}
La comptencia iniciará con la primer prueba de la categoria de 9 y mayores, en seguida empezará la segunda prueba de 9 y mayores. Al termino se nadará la primer y segunda prueba de la categoria 8 y menores para terminar el día con la tercer y cuarta prueba de 9 y mayores.  El segundo día se iniciará la misma dinámica con las pruebas correspondientes. Usando el caso de la tabla \ref{table:orden_example}, el orden de cada prueba sería el siguiente:
\begin{table}[H]
    \centering
    \begin{tabular}{ccc}
        \hline
        Orden del día & Día 1              & Día 2               \\  \hline
        1             & 9 y mayores - 50P  & 9 y mayores - 50P   \\
        2             & 9 y mayores - 100L & 9 y mayores - 100CI \\
        3             & 8 y menores - 25M  & 8 y menores - 25D   \\
        4             & 8 y menores - 25P  & 8 y menores - 25L   \\
        5             & 9 y mayores - 100D & 9 y mayores - 50L   \\
        6             & 9 y mayores - 50M  & 9 y mayores - 100D  \\ \hline
    \end{tabular}
    \caption{Organización de la pruebas siguiendo la organización 2 y el caso de la tabla \ref{table:orden_example}.}
    \label{table:organization2}
\end{table}

\subsection{Tiempo entre heats}
Para lograr una mejor optimización del tiempo total  de la comptencia se propone que entre cada heat exista un tiempo determinado dependiendo unicamente distancia de nado. Los tiempos para cada distancia se encuentran en la tabla  .
\begin{table}[H]
    \centering
    \begin{tabular}{cc}
        \hline
        Distancia (m) & Tiempo (min) \\ \hline
        25            & 2            \\
        50            & 2            \\
        100           & 3            \\  \hline
    \end{tabular}
    \caption{Tiempo entre heats dependiendo de la distancia de nado}
    \label{table:tiempos}
\end{table}