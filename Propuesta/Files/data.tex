\hrule
\section{Datos de cada nadador}
Los datos seran petidos a los responsables de cada organización. La manera de organizar los datos sera por medio de un archivo con extension csv. El formato de este debera contener los siguientes elementos:
\begin{itemize}
    \item Nombre(s) del nadador.
    \item Apellido materno.
    \item Apellido paterno.
    \item Edad.
    \item CURP.
    \item Primera prueba a nadar.
    \item Segunda prueba a nadar.
\end{itemize}
El curp será necesario para identificar a cada nadador en las estádisticas y prevenir una posible mezcla de datos entre dos o más personas. No será necesario entregar un documento oficial. El archivo con el formato será entregado a cada responsable y será devuelto con la información de sus nadadores. El formato puede ser descargado en el siguiente \href{https://github.com/giovannilopez9808/Competencia_2021/raw/main/Formats/data.zip}{link}.
\subsection{Formato para el registro de las pruebas}
Las columnas \textbf{Prueba 1} y \textbf{Prueba 2} deberán de respetar el formato establecido en la tabla \ref{table:codigos_de_pruebas}.
\begin{table}[H]
    \centering
    \begin{tabular}{ll}
        \hline
        \textbf{Nombre de la prueba} & \textbf{Código de registro} \\ \hline
        25 metros libre              & 25L                         \\
        25 metros dorso              & 25D                         \\
        25 metros pecho              & 25P                         \\
        25 metros mariposa           & 25M                         \\
        50 metros libre              & 50L                         \\
        50 metros dorso              & 50D                         \\
        50 metros pecho              & 50P                         \\
        50 metros mariposa           & 50M                         \\
        100 metros libre             & 100L                        \\
        100 metros dorso             & 100D                        \\
        100 metros pecho             & 100P                        \\
        100 metros combinado         & 100C                        \\ \hline
    \end{tabular}
    \caption{Códigos de registro de las pruebas a competir.}
    \label{table:codigos_de_pruebas}
\end{table}
\subsubsection{Ejemplo}
\begin{table}[H]
    \centering
    \begin{tabular}{ccccccc}
        \hline
        \textbf{Nombre} & \textbf{AM} & \textbf{AP} & \textbf{Edad} & \textbf{CURP}      & \textbf{Prueba 1} & \textbf{Prueba 2} \\
        \hline
        Giovanni        & López       & Padilla     & 22            & LOPG980813HCLPDV02 & 50L               & 100D              \\
        \hline
    \end{tabular}
    \caption{Ejemplo del registro de datos en el formato.}
    \label{table:format_example}
\end{table}
